\documentclass[11pt,landscape]{article}
\usepackage[a3paper]{geometry}
\usepackage{graphicx}
\usepackage{fancyhdr}
\usepackage{multirow}
\usepackage{multicol}
\usepackage{textcomp}
\usepackage{gensymb}
\usepackage{amsmath, amssymb}
\usepackage{float}
\usepackage{wrapfig}
\usepackage{hyperref}
\usepackage[parfill]{parskip}
\usepackage{glossaries}
\usepackage{amsmath}
\usepackage{mdframed}
\usepackage{caption}
\usepackage{siunitx}
\geometry{margin=2cm}

\graphicspath{{./images/}}

\pagestyle{fancy}
\setlength{\headheight}{24pt}
\fancyhead[L]{GDP Group}
\fancyhead[R]{Design of Smart Autonomous Golf AGC}
\fancyfoot[C]{\thepage}


\title{GDP Draft Report}
\author{GDP Group}



\begin{document}
\pagenumbering{arabic}
\maketitle
\newpage
\begin{multicols}{3}
\tableofcontents

\section{Introduction}
In this modern era, mobile robot platforms are not strange concepts and have
been utilised not only for transportation but also have been diversified in
other services such as medical, industrial and even sports. Autonomous golf
caddies are also emerging technologies in the sport; where they provide support
by carrying their golf clubs and provide useful information such as weather
forecast and wind trajectory throughout their round of golf. There are typically
two modes of operation provided by current golf caddies where it can either
follow the golfer autonomously though some type of tracking sensor is required
such as a Bluetooth remote control, or through man powered control where the
golfer would need to manually guide the caddy through the golf course.
\cite{choi_2020}. These modes are of course interchangeable with the transition
from manual to autonomous and vice versa made via an interaction between the
golfer and the caddy. Recent improvements on the design concepts of autonomous
golf caddies on the market have been made especially on the autonomous
following. The M5 GPD DHC Trolley for instance, features a rechargeable lithium
battery and secure Bluetooth connectivity remote control that enables the caddy
to follow the golfer independently as long as the remote control is on-person
and within 50 meters of the caddy \cite{golf_2022}.

However, such high-end technology leads to the product being very expensive
where this particular product cost £950 thus not being accessible to the general
public. In addition, the caddy although can be independent, would need some
guidance as to where it should go considering that the golf course is full of
hazards such as bunkers and lakes. Besides that, though current technologies are
good and helps the golfer in ways as previously mention, there is a lot of
potential in improving the caddy. Experienced golfers would have better
understanding of their rounds and would have better intuitions to make decisions
on which golf clubs that would be best used for the current round. However,
beginner golfers do not have that luxury and there is potential there to have
the caddy aid them in making decisions based on computational analysis of their
rounds, weather forecast and wind trajectory. 

Improvements can be made to the current autonomous golf caddies so that it would
be more affordable and accessible to the general public while aiding current
golfers with their performance and encourage new talents to the sport. This
project considers the design, manufacture and test of a prototype Autonomous
Golf Caddy (AGC), that emphasise on these improvements while maintaining the
architecture and integrity of current golf caddies following the required rules
and regulations. Although current regulations do not permit computer aids for
professional games as this would be an unfair advantage, the improvements on the
information feature could be used during casual games and training so that
golfers could analyse their performance by comparing it to computational
analysis of their games, weather forecast and wind trajectory.  

\subsection{Aims}
The idea of this project was proposed by our client, an avid golfer who wishes
to improve his game of golf and commercialising a new product to the market
targeting new talents in golf. Based on the clients requirements and considering
the requirements of the GDP, the team has set a list of aims prior to the start
of the Group Design Project, aligning both stakeholder's requirements, to
achieve at the end of the project: 
\begin{itemize}
    \item Develop a prototype of a smart autonomous golf caddy within the budget
    provided by the university that is able to follow the golfer around the golf
    course while avoiding unallowed locations such as the greens, bunkers and
    waters. 
    \item Design the caddy that could offer golf clubs to the golfer according
    to the golfer`s needs 
\end{itemize}

\subsection{Objectives}
After understanding the ideas proposed by our client, the team has identified
several objectives that needs to be fulfilled in order to achieve the aims of
this project: 
\begin{itemize}
    \item Design and implement a system that allows autonomous tracking between
    a golfer and the caddy 
    \item Develop and implement machine learning algorithm into the caddy that
    will be used to record data from the golfer’s performance 
    \item Design and implement a system that will give out the optimum golf club
    needed for the next shot based on the computer aided algorithm 
    \item Design and develop a User-Interface system where the golfer is able to
    communicate with the caddy of their needs 
\end{itemize}

\subsection{Resources}
The team was allocated the baseline budget from the university of £850 and was
also supported by our client where the team is provided with golf equipment to
design the golf caddy and conduct testing. A full break down of the project our
fundings can be seen in Table X. 

\subsection{Team Roles and Work Breakdown}
The team was structured and organised in a way that everyone`s skills and
experiences on similar works that could make the best out of this project. The
team`s breakdown of roles can be seen in Table X 

\subsection{Ethical and Sustainability Considerations}
The aim of the project is to design and test the prototype to improve the caddy
by implementing new features to aid a golfer’s performance at an affordable
commercial price. An ethical concern pertains to whether or not the caddy is
follow the rules and regulations of standard golf. Although the implementation
of the machine learning algorithm is not allowed as this would be an unfair
advantage in a professional game, the intention of this feature is focused on
training and encouraging new talents that are keen in joining the sport. Other
than that, all of designs and implementations are acceptable within the rules
and regulations. In terms of sustainability, the usage of a rechargeable lithium
battery encourages greener alternative to energy consumption. Besides that, the
design of the caddy follows standards of current products on the market with
slight adjustments in the dimensions which does not have any significant impact
to the golfer, people around and its surrounding. Since the implementations of
the design and systems are on an existing caddy, this have saved the team a lot
of material resources and cost.  

\subsection{Design Process}
The smart autonomous golf caddy consists of many sub-systems thus requires a
structured and clear design process in order to be successful. The design
process covers all the stages of the design and manufacturing process that the
team followed in order to achieve a reliable and high-quality prototype. 

All components of the caddy including its systems were interconnected thus a
design process flowchart as seen in Figure X, was done to ensure all the design
process was clear and done properly to obtain the best result for the project.
The design process follows a 5 stage approach. The first stage consist of
defining the problem, brainstorming possible solutions and exploring new
concepts and possibilities for the design of the project. It was important for
the team to cater to the requirements of the client without going astray of the
requirements of the GDP. Initially, the request was to design and develop a
fully functional smart autonomous golf caddy that was commercialisable. However,
this was not possible considering the time and budget allocated for this project
by the GDP stakeholders thus the team proposed to design and develop a
functioning prototype that could be further improved in the future. Next, the
team identified the design specifications of the standard caddy, research and
explore new features and concepts, and subsequently evaluated and compared the
ideas with current products available on the market. 

The second stage was to select an approach and develop a design proposal. This
stage consist of the team working on their roles, introduce the concepts to the
team and client. Approaches that did not meet the requirements were rejected and
the design proposal were refined during discussions. It was also important to
the team that the client were presented with all the information and ideas in
order to meet the client’s requirements. It was also found very insightful and
helpful as the client gave feedback and new ideas to consider. 

The third stage was simulation and prototyping. The algorithms and codes for the
machine learning and golf caddy following were develop and simulated. Based on
the design specifications, the mechanical designs, motor controls and
electronics were tested. All components of the project were simulated or tested
individually to ensure that individual part problems were reduced to a minimum
before assembly. This process was important to mitigate any risk that could be
hazardous to the team and everyone involved during the prototyping process and
increase the efficiency of using the budget to avoid wastage. 

The fourth stage is assembly and testing. After assembling all individual
components to make the prototype caddy, testing was conducted at Southampton
Municipal Golf Course. The testing involved the client, as an experienced
golfer, to observe and analyse the functioning of the prototype caddy in a real
golf course setting. The fifth and final stage is to identify further
improvement that could be made to the design. In the case of this project, this
could mean developing and improving the design to commercialising standards
given the necessary time and funding. 


\newpage
\end{multicols}
\begin{multicols}{2}


    \newpage
    
\section{Mechanical Design}
    In this section, several mechanical components that were designed,
    manufactured and implemented are discussed. However, considering the the
    grandscale of things, this project only places its proposed design in the
    prototype phase of this novel idea. Therefore, an emphasis was placed on
    achieving the most effective design that will achieve the objectives set out
    at the start of this project whilst making best use of the resources
    available such as manufacturing workshops, material and the inherited
    components from the golf caddy. 
    
    \subsection{Release Mechanism}
    One of the key objectives for this project was to design and implement a
    release mechanism that will raise a golf club as a form of suggestion for
    the golfer’s next shot towards the green, based on a machine learning
    algorithm. This is a novel idea presented by our project advisor and is one
    of the many innovative designs that will be implemented in this project.
    
    \subsubsection{Design Requirement}
    Initially, the workspace that will hold such mechanism had to be identified
    as it will be part of the design requirements before making any prototypes.
    Three feasible locations were considered which were to either be integrated
    in the golf bag, to be integrated on the golf caddy such that the golf bag
    will rest on top of it, or to be in a separate compartment that will need to
    be attached to the golf bag and golf caddy. The selected option was for the
    mechanism to be integrated in the golf bag as this reduces the likeliness of
    errors occurring since the mechanism does not have to be removed and
    reconnected after every game from the golf bag. The design requirements are
    listed as follows:
    
    
    \begin{itemize}
    \item Need to lift at least 10 cm above other golf clubs to make it clear
    which club is being suggested
    \item Needs to withstand the weight of the golf club which produces an axial
    loading
    \item Needs to be lightweight since the golf bag will be carried by the user
    before it is mounted on the caddy
    \item Needs to be a robust design since it will be used multiple times
    during a game
    \item Needs to fit inside the designated workspace that is inside the golf
    bag
    \end{itemize}
    
    \subsubsection{Concept Design}
    Various mechanism that provides linear translation motion, which are similar
    to what is desired, were explored. However due to the design requirements,
    two main concept ideas were considered to be the most suitable and they are
    shown in Figure \ref{fig:design}.
    
    \begin{figure}[H]
        \begin{center}
            \includegraphics[width=0.5\textwidth]{Figure19.jpg}
            \captionof{figure}{Concept design for release mechanism}
            \label{fig:design}
        \end{center}
    \end{figure}
    
    
    Idea 1 consisted of an umbrella type mechanism which uses a compression
    spring, and with the use of a stepper motor and cam, the system can be
    released from a “loaded” state to elevate the component. Idea 2 was more
    innovative as it made use of the bolt and nut analogy. A plate holder is
    attached to a nut which is placed on a stepper motor with a threaded rod.
    This plate holder is placed within a tube and by using the inner wall of the
    tube as a restriction to prevent the rotation of the nut when the stepper
    motor provides a torque, the plate holder’s only direction of motion can be
    either up or down the tube. Idea 2 was selected as the preferred design of
    choice due to various reasons. The most important reason was that this
    design does not require the user to push down the golf club in order to put
    the system in a “loaded” state compared to Idea 1. It also offers a safer
    and more controlled way to elevate the golf club compared to the spring
    action which will make the golf club jump upon release. Another benefit was
    that there are less parts involve in order to make the whole mechanism
    compared to Idea 2 and therefore there is a less chance for failure to
    occur. To calculate the required torque for the stepper motor, the maximum
    weight of a golf club that was measured to be 700g, was used in further
    calculations. The torque was calculated, using a pulley analogy, with the
    formula shown below.
    
    \newpage
    
    \begin{center}
        \begin{equation}
            F = W = mg
            \label{eq:raising_force}
        \end{equation}
    \end{center}
    \begin{center}
        \begin{equation}
            T = Fr = Mgr
            \label{eq:raising_torque}
        \end{equation}
    \end{center}
    
    m is the mass of the golf club, g is the gravitational acceleration
    constant, and r is the distance from the centre of the plate holder (the
    assumed position of the golf club’s handle) to the centre of the threaded
    rod. A torque value of 0.13 Nm was calculated, however, the closest rated
    torque for a motor that could be obtained was the NEMA 17 stepper motor
    which has an output of 0.42 Nm. This motor was chosen since it allows for a
    safety factor of 3.2 This would allow for a safety factor of 3.2 which is
    sufficient considering the previous calculations neglected the friction
    acting between the nut and the threaded rod as well as the fiction between
    the plate holder and the inner wall of the tube. 
    
    
    \subsubsection{Testing and Improvement}
    Idea 2 was then made into a prototype, as shown in Figure \ref{fig:release},
    to test the effectiveness and robustness of the system. 
    
    \begin{figure}[H]
        \begin{center}
            \includegraphics[width=0.5\textwidth]{Figure20.jpg}
            \captionof{figure}{Release Mechanism Prototype}
            \label{fig:release}
        \end{center}
    \end{figure}
    
    The initial testing of the prototyping demonstrated that the design worked
    well, with a time of 12.7 seconds recorded to lift the golf club by 10 cm.
    However, a serious issue was discovered where the plate holder can get stuck
    within the tube and thereby failing the whole system. Furthermore, there was
    bending observed on the plate holder when the golf club was not perfectly
    placed in the middle of the plate. Another issue was that the time taken for
    the club to be raised was particularly long which is due to the pitch of the
    threaded rod being short. In consequence of that, some improvements were
    made to this design. The stepper motor with the threaded rod was changed to
    one with a leadscrew integrated as its shaft and this will decrease the time
    to raise the golf club.  A way of mitigating the risk for a system failure
    due to the plate holder getting stuck inside the wall of the tube, was to
    use a cylinder where its lower plate will have the leadscrew nut mounted on
    using nuts and bolts. It was deemed to be a more robust design that can
    easily support the weight of the golf club and with the implementation of
    guiding rails, which are threaded rods in this case, the inner tube walls
    will not be used to restrict the rotating motion of the cylinder. This tube
    was manufactured using 3D printing with polylactic acid (PLA) filament as
    the material. Taking into account the stair stepping effect and porosity
    from 3D printing, the cylinder was made to be sufficiently strong enough to
    hold the maximum weight of a golf club recorded. After incorporating these
    improvements, the calculated time to raise the golf club by 10 cm was
    recorded to be 3.6 seconds which is almost 3.5 times faster than the
    prototyped version. The overall weight of this design only amounts to 400g
    per mechanism and since only three release mechanism has been promised to be
    implemented in this project, the total added weight amounts to 1.2 kg which
    is only an 8\% weight increase from an original golf bag weight of 15 kg. 
    
    
    \subsubsection{Final Design}
    The final and proposed design assembly and the built design is shown in
    Figure \ref{fig:release1}
    
    \begin{figure}[H]
        \begin{center}
            \includegraphics[width=0.3\textwidth]{Figure21.jpg}
            \captionof{figure}{Proposed Release Mechanism (a)Assembly (b)Manufactured Product}
            \label{fig:release1}
        \end{center}
    \end{figure}
    
    
\end{multicols}
\newpage
\begin{multicols}{3}
    \subsection{Golf Bag}
    Golf bag are the essential tools used by the golfer. In this project, the
    golf bag is the core of the entire club raising mechanism. All the mechanism
    and electronics are installed on the golf bag. The golf bag is well designed
    to ensure the entire mechanical and electrical system had sufficient work
    space to operate. 
    
    \subsubsection*{Design Requirement}
    \begin{itemize}
        \item Need to fit all 14-golf club mechanism
        \item Mount mechanical compartment and electrical compartment
        \item Components able to stay still during motion
        \item Compartment need to be easily accessible for maintenance
        \item Lightweight 
    \end{itemize}
    
    \subsubsection*{Design Process}
    The design process started by thinking a feasible release mechanism before
    working on the golf bag. Once the release mechanism as shown above had been
    decided.The golf bag provided by James are dismantled to further understand
    its infrastructure. In order to corporate the release mechanism in the golf
    bag, a new top base design are required. We realized the segregation of the
    golf bag didn't’t reach the bottom of the golf bag (Figure \ref{fig:cloth}
    ). The segregation took place on the frame , a small part of the top part
    are segregated with cloth giving a misconception of all the 14 golf clubs
    are well segregated before dismantling the bag.A segregation is essential to
    separate the release mechanism system and to prevent the golf clubs from
    tangling each other at the bottom of the golf bag. Therefore, tube is
    implemented for each golf clubs so the golf club may not interact with each
    other. Exploration of different material of tube had been done. PVC tube had
    been chosen as it is cheap, accessible and lightweight. The average diameter
    of the golf club is 28mm therefore the inner diameter of the tube must be
    greater than 28mm. The inherited design is unsymmetrical and had
    inconsistent spacing (Figure \ref{fig:top})for all the 14 holes. By removing
    the inherited design will provide more freedom and potential to fully
    utilized the available space of the golf bag. Calculation of the dimension
    had been made by scanning the top frame using a 3D scanner(Figure
    \ref{fig:scan}). It is a innovative and efficient method to extract accurate
    data from irregular object due to its non-linear curvature.
    \begin{figure}[H]
        \begin{center}
            \includegraphics[width=0.18\textwidth]{Figure15.jpg}
            \captionof{figure}{Scan Machine}
            \label{fig:scan}
        \end{center}
    \end{figure}
    
    \begin{figure}[H]
        \begin{center}
            \includegraphics[width=0.2\textwidth]{Figure3.jpg}
            \captionof{figure}{Inherited Top Base Side View}
            \label{fig:cloth}
        \end{center}
    \end{figure}
    
    \begin{figure}[H]
        \begin{center}
            \includegraphics[width=0.2\textwidth]{Figure2.jpg}
            \captionof{figure}{Inherited Top Base Frame}
            \label{fig:top}
        \end{center}
    \end{figure}
    
    
    
    By applying a fill factor of 80\%, it is feasible for each tube to have a
    maximum of 57mm outer diameter. Based from the value, we decided a PVC tube
    with 42mm outer diameter and 38mm inner diameter would provide ease of
    placing the golf clubs into the tube. Using the given dimension, design had
    been made for the top base of the golf bag.
    
    The design of the bag will follow along the PVC tube. With a solid
    foundation for the 14 PVC tubes, the changes made in the release mechanism
    wouldn’t affect the golf bag design as long the release mechanism are
    installed in the dedicated PVC tube. Each tube had its own name as specified
    golf clubs will be placed in it. By doing this, the machine learning will
    release the mechanism of the specified club to the golfer based on its
    decision with the acquired data from the golfer. 
    
    The next step is to design a support that could hold all the 14 PVC tubes.
    Mechanical and electrical compartment had been implemented to ensure the
    mechanical components are undisturbed from the electrical components. The
    bag needs to be extended due to the implementation of the release mechanism
    in the golf bag. The released golf clubs may injure the user if the golf bag
    isn’t extended as the extended golf clubs are in an angle facing the user.
    The golf bag will be unstable if the golf bag isn’t extended as the weight
    distribution are too far from the center of mass of the golf bag. Next, a
    mechanical and electrical compartment had been designed and implemented at
    the bottom of the golf bag. Both designs are simple and accessible to the
    user for future maintenance. After that, a new top base of the golf bag had
    been designed to replace the old unsymmetrical top base as the new design
    provide more freedom of utilizing maximum space in the bag. Once the main
    frame of the golf bag is manufactured, the mechanical mechanism and
    electronics will be installed. 
    
    
    \subsubsection{Golf Bag innovation}
    The golf bag is another one of our innovative creations as we manage to
    create a fully-functioning golf bag with extremely low cost. Low cost is
    achieved by reusing recycled items and using basics resources such as metal,
    wood and acrylic. With manufacturing limitation, cloth and plastic had been
    replaced by metal to extend the golf bag. Wood is used as the support for
    the release mechanism while acrylic is used for the new top base design of
    the golf bag. The golf bag is able to fulfil the design requirements and
    tested to be working well with the mechanical and electrical components.
    
    \subsubsection{Golf Bag Infrastructure}
    The whole bag had 3 different parts, the top base of the golf bag, the
    mechanical compartment and the electrical compartment as shown in Figure
    \ref{fig:golfbag} . The top base of the golf bag act as a support and
    segregation compartment for the golf clubs. The mechanical compartment
    consists of the motor compartment and the release mechanism. The motor is
    placed between two support to ensure the motor keep in place during
    operation. The release mechanism is above the motor compartment. It had a
    support base to ensure the tube and the release mechanism stay aligned. All
    the electronics and the cabling are placed in the electrical compartment to
    ensure the cabling do not disturb the mechanical system and also for the
    ease on waterproofing all the electronics.
    
    
    \begin{figure}[H]
        \begin{center}
            \includegraphics[width=0.4\textwidth]{Figure16.jpg}
            \captionof{figure}{Infrastructure of Golf Bag}
            \label{fig:golfbag}
        \end{center}
    \end{figure}
    
    
    
    \subsubsection{Top base of Golf Bag}
    The top base of the golf bag had been through with 2 manufacturing process
    as shown in Figure \ref{fig:change} . Both design are able to fit 14 equal
    size release mechanism for the golf club. Each slot had equal spacing and
    sufficient space between the slots. PVC tube fit through each slot
    perfectly. Design 1 had been initially laser cut with Plywood before sending
    for 3D printing. The tube fitted the design and are well supported by it.
    Due to the limitation of the 3D printer size (250 x 210 x 210mm), the new
    design had a dimension of (255 x 230 x 41 mm). Therefore it is not feasible
    to 3D print the entire design.
    
    Decision had been made to had the new design installed on the old golf bag
    top frame which is more innovative as the frame could be reused. The new
    design is drawn on Solidworks with the scanned CAD model of the top
    frame(Figure \ref{fig:CAD}). The final design is laser cut with acrylic and
    mounted on the frame with epoxy. The final top base is sprayed with black
    spray to improve aesthetic.
    \begin{figure}[H]
        \begin{center}
            \includegraphics[width=0.18\textwidth]{Figure.png}
            \captionof{figure}{Design Process of Top Base}
            \label{fig:change}
        \end{center}
    \end{figure}
    
    \begin{figure}[H]
        \begin{center}
            \includegraphics[width=0.2\textwidth]{Figure.jpg}
            \captionof{figure}{CAD model of Top Base}
            \label{fig:CAD}
        \end{center}
    \end{figure}
    
    
    \subsubsection{Extension of Golf Bag}
    The golf bag is extended to fit the release mechanism. The release mechanism
    contributes extra 138mm vertical height of the golf bag. The placing of the
    release mechanism is crucial as it may affect the overall performance. Due
    to the additional vertical height, the golf bag is extended to compensate
    the height from the release mechanism. Without the extension, the released
    golf clubs may injure the user since the golf clubs are released at a
    direction toward the user. The bag had been extended using rolled metal
    sheet. Calculation had been made for the circumference of the metal sheet.
    The metal sheet is spot welded to make a perfect circle. The golf bag is
    extended by 150mm. 100mm are for the release mechanism, 38mm for the motor
    and 12mm for the electrical compartment.
    
    \subsubsection{Mechanical Compartment}
    The mechanical compartment consists of the mechanism and the stepper motor.
    As shown in Figure \ref{fig:SIDE} and \ref{fig:TOP}, the mechanism are
    installed on top of the stepper motor. The stepper motor is placed on the
    electrical top base. A motor base had been added to secure the motor between
    the motor base and the electrical base. The release mechanism is mounted on
    the motor base to ensure the mechanism stay stationary in the bag. The
    support base provide support to the PVC tubes and ensure the release
    mechanism remained in a vertical direction.
    
    \begin{figure}[H]
        \begin{center}
            \includegraphics[width=0.3\textwidth]{Figure6.jpg}
            \captionof{figure}{Side view of Mechanical Compartment}
            \label{fig:SIDE}
        \end{center}
    \end{figure}
    
    \begin{figure}[H]
        \begin{center}
            \includegraphics[width=0.2\textwidth]{Figure7.jpg}
            \captionof{figure}{Top view of Mechanical Compartment}
            \label{fig:TOP}
        \end{center}
    \end{figure}
    
    \subsubsection{Electrical Compartment}
    The Electrical Compartment consist of a wooden laser cut top base and the
    inherited golf bag bottom base. The wooden laser cut base had 14 small holes
    on it as shown in Figure \ref{fig:ELECBASE}. The main function of the holes
    is to allow the electrical cable from the mechanical compartment to connect
    the electrical components.
    
    \begin{figure}[H]
        \begin{center}
            \includegraphics[width=0.2\textwidth]{Figure8.jpg}
            \captionof{figure}{Electrical Top Base}
            \label{fig:ELECBASE}
        \end{center}
    \end{figure}
    
    
    
    \begin{figure}[H]
        \begin{center}
            \includegraphics[width=0.3\textwidth]{Figure9.jpg}
            \captionof{figure}{Electrical Top Base CAD}
            \label{fig:ELEC}
        \end{center}
    \end{figure}
\end{multicols}

\begin{multicols}{2}
    The location of the holes (Figure \ref{fig:ELEC}) is placed beside the
    mechanical component for the ease of the cable to travel to the electrical
    compartment. The holes are also designed to ensure none of the mechanical
    component will obstruct the hole. The electrical compartment consists of the
    cabling from the motor and a main cable which connects to the caddy. All the
    electronics are placed in the electrical compartment for ease of water
    proofing and to ensure the cabling do not disturb the mechanical system. 
    
    \subsubsection{Installation of release mechanism on golf bag }
    The support base had been first mounted on the golf bag with L bracket. Once
    the extension of the golf bag are complete. the release mechanism are
    installed on the motor base and mounted on the circular metal sheet with L
    bracket. The release mechanism are installed into the PVC tube and the
    electrical top base are mounted below the stepper motor using L bracket.
    Finally , the inherited bottom base cover are placed below the circular
    metal sheet. Toggle latches are installed on both the circular metal sheet
    and the bottom cover to achieve easy accessible for the user. The process is
    shown in Figure \ref{fig:installation}.
    \begin{figure}[H]
        \begin{center}
            \includegraphics[width=0.4\textwidth]{Figure30.jpg}
            \captionof{figure}{Release Mechanism installation process in golf bag}
            \label{fig:installation}
        \end{center}
    \end{figure}
\end{multicols}

\newpage

\begin{multicols}{3}
    \subsection{LCD screen casing}
    In this project, a LCD screen is implemented which will have a GUI for data
    representation.A LCD screen casing had been manufactured to protect the LCD
    screen from environmental damage such as rainwater and physical debris
    damage. The overall casing design are shown in Figure \ref{fig:casing}.
    Different location of mounting the LCD screen on the caddy had been
    explored, The best location of mounting the LCD screen on the caddy will be
    right below the caddy handle as it can be easily viewed by the user nearby
    the caddy. The location do not affect the the folding operation of the caddy
    (Figure) as it utilized the abundant space of the caddy. 
    
    CADDY FOLDED WITH SCREEN
    
    A 1 m electrical ribbon had been used to connect the LCD screen with the
    raspberry pi located on the electrical housing of the caddy. The electrical
    ribbon is heat shrink to provide protection for the ribbon and water
    resistance. The ribbon travel from the electrical housing of the caddy to
    the LCD screen through the caddy inner tube pathway as demonstrated in
    Fig.(dlsd) main frame with ribbon

    \begin{figure}[H]
        \begin{center}
            \includegraphics[width=0.2\textwidth]{Figure10.jpg}
            \captionof{figure}{Casing Design}
            \label{fig:casing}
        \end{center}
    \end{figure}
    
    \subsubsection{Screen Holder}
    As shown in Figure, the screen holder is mounted on the back of the bottom
    casing. The screen holder is designed to allow the LCD screen to sit firmly
    on the screen holder (Figure \ref{fig:holder}). A cylinder metal rod (Figure
    \ref{fig:rod})had been added in between of the screen holder to ensure the
    screen holder do not rotate more than 90 degree.
    
    \begin{figure}[H]
        \begin{center}
            \includegraphics[width=0.2\textwidth]{Figure14.jpg}
            \captionof{figure}{LCD screen with Screen Holder}
            \label{fig:holder}
        \end{center}
    \end{figure}
    
    \begin{figure}[H]
        \begin{center}
            \includegraphics[width=0.2\textwidth]{Figure31.jpg}
            \captionof{figure}{Screen holder with metal rod support}
            \label{fig:rod}
        \end{center}
    \end{figure}
    
    
    
    \subsubsection{Friction Hinge}
    The screen can be fold and unfolded by 90 degrees using a friction hinge
    concept (Figure\ref{fig:unfold} and Figure \ref{fig:fold}). The screen
    holder is designed to operate along with the friction hinge operation. The
    friction hinge relies on the friction provided from the nuts and bolts. By
    providing sufficient friction, the screen could be fixed and folded easily.
    
    \begin{figure}[H]
        \begin{center}
            \includegraphics[width=0.2\textwidth]{Figure17.jpg}
            \captionof{figure}{Unfold LCD screen}
            \label{fig:unfold}
        \end{center}
    \end{figure}
    
    \begin{figure}[H]
        \begin{center}
            \includegraphics[width=0.2\textwidth]{Figure18.jpg}
            \captionof{figure}{Folded LCD screen}
            \label{fig:fold}
        \end{center}
    \end{figure}
    
    
    \subsubsection{Bottom casing}
    The bottom casing is designed according to the needs of installation and the
    electrical need. The LCD screen fit the bottom casing perfectly as shown in
    Figure \ref{fig:LCD}. From Figure \ref{fig:bottom}, there is a slot to allow
    connection for the electrical ribbon from the screen to the caddy. Bolts are
    used to screw directly on the LCD screen through hole 1 and hole 2. While
    the bolts on hole 3 and 4 will go through the screen holder before screwing
    on the LCD screen. Therefore, the bottom case will mount firmly to the
    screen holder.
    
    \begin{figure}[H]
        \begin{center}
            \includegraphics[width=0.3\textwidth]{Figure32.jpg}
            \captionof{figure}{Front view of LCD screen fit in casing}
            \label{fig:LCD}
        \end{center}
    \end{figure}
    
    
    \begin{figure}[H]
        \begin{center}
            \includegraphics[width=0.3\textwidth]{Figure11.jpg}
            \captionof{figure}{Back view of Bottom casing}
            \label{fig:bottom}
        \end{center}
    \end{figure}
    

    \subsubsection{Top casing}
    The top screen covers the bottom casing edges and the LCD screen to ensure
    no water flow to the internal screen casing. A screen protector is installed
    on the large empty slot of the top screen. The selected screen protector is
    made of temper glass which doesn’t affect the touch screen functionality.
    The screen protector is glued on the casing and a layer of resin is added to
    prevent water intake. The screen protector provides the ability of using the
    LCD screen during rainy condition as it is well water-proofed. The top
    casing are mounted with the bottom casing with nuts and bolts through Hole
    C,D,E,F as shown in Figure \ref{fig:HOLE} This ensures the entire casing do
    not drop from the screen holder during motion.
    
    \begin{figure}[H]
        \begin{center}
            \includegraphics[width=0.3\textwidth]{Figure13.jpg}
            \captionof{figure}{Top Casing with holes}
            \label{fig:HOLE}
        \end{center}
    \end{figure}
\end{multicols}

\newpage
\begin{multicols}{2}
    \subsection{CADDY ELECTRICAL BASE}
    From the inherited golf caddy, there were close to no electrical systems
    since only the wheels were motorised. However, with the addition of all the
    newly added electrical systems from this project, an electrical base was
    required in order to house all the electrical circuits that were made. The
    list of components that will be in the electrical base are the EMG49 motors,
    MD49 driver, Raspberry Pi, GPS module, battery regulation circuit and
    release mechanism circuit.
    
    \subsubsection{Concept Design}
    After investigating the inherited golf caddy for possible locations that can
    be used to have the electrical base, it was determined that the only
    feasible place was where the previous battery located. After removing the
    previous motors and battery plate holder, the new allocated space was
    measured and built on SOLIDWORKS in order to model different possible design
    for the electrical base. Figure \ref{fig:base} shows the available space
    from the caddy after the motors were removed.
    
    \begin{figure}[H]
        \begin{center}
            \includegraphics[width=0.4\textwidth]{Figure22.jpg}
            \captionof{figure}{Location for electrical base}
            \label{fig:base}
        \end{center}
    \end{figure}
    
    Three possible designs were initially made on CAD and they are shown in
    Figure \ref{fig:box}.
    
    \begin{figure}[H]
        \begin{center}
            \includegraphics[width=0.5\textwidth]{Figure23.jpg}
            \captionof{figure}{Concept design for electrical base}
            \label{fig:box}
        \end{center}
    \end{figure}
    
    
    Idea 3 was chosen as the best option since it allows for an easy access to
    the electrical components which is beneficial for maintenance and fault
    checking should any issues occur. It also allows the user to easily remove
    and charge the battery when required without interfering with any of the
    electrical circuits along the way and it provides a better airflow compared
    to being clustered in the “box” compartment. It is also beneficial for the
    centre of mass of the caddy as the battery, which is the heaviest amongst
    the component on the electrical base, is as close as possible to the centre
    of the caddy. It allows for a more stable drive whilst minimising the chance
    of the caddy toppling over when going on a hill. The combined weight of all
    the components that will be in the base was determined to be about 7 kg.
    Based on this and the fact that this base is only being used for testing
    since this project is at the prototype stage, it was determined that 6 mm
    birch plywood was the best material of choice. It is cheap, easily
    accessible and can provide enough strength to hold all the components as
    well as being lightweight enough to not have an impact on the motors. Laser
    cutting was used as the manufacturing process for time saving and due to the
    possibility of employing box joints as well as mortise and tenon joints in
    order to easily assemble the design. The joints were further reinforced
    using gorilla wood glue for added reassurance of the structural integrity of
    the design.
    
    
    
    
    \subsection{POWERTRAIN AND WHEELS}
    The powertrain on the inherited golf caddy consisted of a motor which was
    powering the rear wheels on a single axle. It allowed the caddy to the
    remotely controlled which was used to assist the user in pushing the caddy
    around. However, a user input was still required in order to navigate the
    caddy, for example when changing direction, since the front wheels were
    attached in such a way that would only allow forward and backward drive.
    Figure \ref{fig:train} shows the drivetrain of the inherited caddy.
    
    \begin{figure}[H]
        \begin{center}
            \includegraphics[width=0.5\textwidth]{Figure25.jpg}
            \captionof{figure}{Inherited Caddy Powertrain}
            \label{fig:train}
        \end{center}
    \end{figure}
    
    Due to the introduction of the autonomous tracking system and release
    mechanism, new motors are required in order to propel the caddy with the
    extra added weight. This powertrain required some changes to be made since
    it was not designed for autonomous driving.
    
    \subsubsection{Design Requirement for Powertrain}
    Two important design changes were identified. Since the caddy will have to
    make turns when required during the autonomous tracking, a speed
    differential between each of the rear wheel is required. Therefore, two
    separate motors are needed to control each rear wheel individually.
    Furthermore, a ball castor type support should be added at the front of the
    caddy in order to facilitate the turning action of the caddy as well as
    providing the ability for the caddy to have its own centre of rotation. 
    
    
    
    \subsubsection{Rear Wheel}
    In order to be cost effective and time efficient instead of manufacturing or
    buying a wheel from a third party seller, the previous rear wheels were
    used. However, these wheels were not compatible to be attached directly to
    the new motors since it required a locking mechnaism that is located on the
    previous axle. Therefore, this axle was cut down at two separate locations
    in order to retain this locking mechanism as well as a bracket that is used
    to securely connect it to the main frame of the caddy. Figure \ref{fig:rear}
    shows the locations that were determined to be the optimum cut off points.
    
    \begin{figure}[H]
        \begin{center}
            \includegraphics[width=0.5\textwidth]{Figure26.jpg}
            \captionof{figure}{Rear axle from inherited golf caddy}
            \label{fig:rear}
        \end{center}
    \end{figure}
    
    Considering the space available, where the wheels need to be mounted, and
    the dimensions of the motors to be used, the connection between the shaft of
    the motor and the axle needs to be close type and secure connection. Since
    welding was not feasible due to the contact surface being too small between
    the motor shaft and the axle, adhesive was used. In order to have more
    contact surface when applying the adhesive, the motor’ shaft was immersed
    inside the axle where it allows for a more robust connection. Figure
    \ref{fig:auto} shows the rear wheels attached to the motors and mounted on
    the base plate of the electrical base.
    
    \begin{figure}[H]
        \begin{center}
            \includegraphics[width=0.5\textwidth]{Figure27.jpg}
            \captionof{figure}{Autonomous driving compatible powertrain}
            \label{fig:auto}
        \end{center}
    \end{figure}
    
    
    \subsubsection{Front Wheel}
    The front wheel was decided that it will be incompatible to be used for
    autonomuos driving and therefore a swivel castor wheel was chosen as its
    best replacement. After removing the intial front wheel and its
    restrictions, the mounting space was at an incline fo 45 degrees. Since a
    flat surface is required to mount the swivel castor wheel, a piece of wood
    which was cut to the appropriate dimension, was in order to fill the gap.
    The wooden piece dimension also took into account the range of rotation of
    the wheel. Figure \ref{fig:castor} shows the process for mounting the swviel
    castor wheel on the caddy. 
    
    \begin{figure}[H]
        \begin{center}
            \includegraphics[width=0.5\textwidth]{Figure28.jpg}
            \captionof{figure}{Swivel Castor Wheel installation process}
            \label{fig:castor}
        \end{center}
    \end{figure}
    The wooden piece was later wrapped in vinyl for aesthetic purposes.
    
    
    \subsubsection{Wheel Assembly}
    Figure \ref{fig:sprayed} shows the wheels mounted on the golf caddy. 
    
    \begin{figure}[H]
        \begin{center}
            \includegraphics[width=0.5\textwidth]{Figure29.jpg}
            \captionof{figure}{Full Wheel installation process}
            \label{fig:sprayed}
        \end{center}
    \end{figure}
    
    
    \subsection{Further Improvement}
    
    For the continuation of this project, to make it a commercially viable
    product, the future team undertaking it should consider the following.
    
    \subsubsection{Release Mechanism}
    For the release mechanism, one of the key improvements that could be
    implemented would be the use of gears for the mechanism as it can save up a
    lot of weight instead of using one stepper motor per golf club. Other
    materials such as acrylic, wood or harder steel should be explored in order
    to be used as the guiding rails since the current one being used are
    threaded rods of 2 mm diameter which can easily bend. This can cause issues
    to the release mechanism’s elevating platform since it can get stuck at the
    point where the bending of the guiding rails occurs. Currently, the way to
    operate the release mechanism, requires the user to connect the cable from
    the golf bag to the caddy’s electrical base. An improvement to that could be
    to implement a circuit within the golf bag that can communicate with the
    circuit on the golf caddy through WIFI or Bluetooth, thus reducing the
    possibility of the connection being loose which would not allow the system
    to work. It also benefits the user since they will not have to worry about
    unintentionally pulling the circuit system inside the electrical base on the
    caddy every time they remove the golf bag form the golf caddy. 
    
    \subsubsection{Golf Bag}
    The current golf bag are designed to be able to implement waterproofing. Due
    to time constraint, the concept had not been implemented but could be done
    easily with the current golf bag. Both Mechanical and Electrical Compartment
    could be sealed with resin to prevent water entering the system. A water
    sink could be implemented above the Mechanical Compartment. A tube could be
    used to release the accumulated water from the water sink.
    
    \subsubsection{Golf Caddy Infrastructure}
    The golf caddy could be more futuristic if the caddy could be redesigned.
    Function of one button folding and unfolding could be achieved if the caddy
    are well designed.
    
    \subsubsection{Caddy's Electrical Base}
    It might be better to explore further designs since the current base is too
    tall and it does not allow the full closure of the caddy when the golf bag
    is removed. The need for waterproofing the “box” compartment which contains
    all the electrical circuits should be implemented in the design stage as one
    of the key design requirements. The current way to close the base is just a
    sliding type of closure and it could be improved by using hinges or some
    other component that can make the base more compact with regards to the
    waterproofing. In terms of material selection, it might be better to use
    acrylic for a commercially viable product since it will look better
    aesthetically. 
    
    \subsubsection{Front Wheel}
    In terms of the current connection of the swivel castor wheel, it might be
    best to make changes to the main frame of the golf caddy instead of using
    the piece of wood to make a flat surface. Moreover, the front wheel itself
    should be considered to change since it might not provide enough grip to be
    used on a golf course for a long period of time.
    
\end{multicols}


\newpage
\begin{multicols}{3}
\section{Software}
\label{software}
In this section all software required for the operating cycle of the AGC is
discussed. The AGC main computer is a Raspberry Pi 4 running a 64-bit Debian
port, the same is true for the Raspberry Pi used in the tracking pod. The GPS
navigation system requires high precision decimal storage to operate properly.
Data type \verb|float32| can store 23 significant bits, compared to data type
\verb|float64| which can store 52 significant
bits\cite{floating_point_goldberg}. The calulcations shown below in Fig.
(\ref{fig:float_calcs}) show the physical significance of which datatype is
chosen.
\begin{figure}[H]
    \begin{mdframed}
        Only $180^{\circ}$ need to be accounted for because sign bit is
        separate, so to represent $180^{\circ}$ the bits required is:
        \begin{center}
            \begin{equation*}
                \log_2{180} \approx 7.49185 bits
            \end{equation*}
        \end{center}
        This leaves the $n - 7.49185$ significant bits left for sub degree
        representation, the physical precision in meters we can then derive from
        each data type of a longitude value at the equator can be calculated as
        shown in the equations below.\newline
        \begin{center}
            \begin{minipage}{0.45\textwidth}
                \begin{mdframed}
                    For \verb|float32| $n=23$:
                    \begin{center}
                        \begin{equation*}
                            \frac{\frac{R_e \cdot \pi}{180}}{2^{\left(52 - \log_2{180}\right)}} \approx 2.386m
                            \label{eq}
                        \end{equation*}
                    \end{center}
                \end{mdframed}
                \end{minipage}
                \begin{minipage}{0.45\textwidth}
                    \begin{mdframed}
                For \verb|float64| $n=52$:
                \begin{center}
                    \begin{equation*}
                        \frac{\frac{R_e \cdot \pi}{180}}{2^{\left(52 - \log_2{180}\right)}} \approx 4.44nm
                    \end{equation*}
                \end{center}
            \end{mdframed}
            \end{minipage}
        \end{center}
        \center The radius of Earth. $R_e$, value used was $6371000m$. 
    \end{mdframed} 
    \captionof{figure}{Calculations showing physics precision of single vs
    double precision floats}
    \label{fig:float_calcs}
\end{figure}
The $180^{\circ}$ is normalised to a power of two so we are able to use the non
integer value of bits in the calculation for physical precision, were we to not
perform this normalisation then $7.4918\rightarrow8$ full bits are required to
represent the degrees and the final precision is slightly less for each
datatype. To ensure this calculated precision is representative of our system
all GPS coordinates are normalised to a power of two. We can see from the
calculations that a significantly higher precision is achieved using double
precision floating point data type to store GPS values. The precision yielded by
the double precision floating point is unnecessary, however the precision of the
single precision floating point is too low as our GPS module is capable of
providing more accurate GPS measurements as discussed in section
\ref{electronics}. Therefore double precision floating point datatypes will be
used to store GPS data, which requires us to use a 64-bit compatible operating
system which is why the 64-bit Debian port was chosen. The higher precision
achieved comes at the cost of slightly higher compte time for calculations,
however as most of the algorithms onboard operate only on small amounts of data,
this effect is unnoticable when operating the AGC.

\subsection{Control Software}
\label{control_software}
This section discusses the software used to make the AGC follow the golfer. All
of the control software apart from the elctronics specific code, such as motor
controller communication, was written first using the OpenGL simulator that was
built throughout the year. Building and using a simulator allowed us to write
and test all the control software / algorithms that would be required to make
the AGC follow the golfer as intended while avoiding hazard regions.

All of the control software tested in the simulator was written in C as this was
easiest to use with OpenGL, however we decided to change to PyQt5 for designing
the onboard GUI. As the ACG software was now to be written in Python the control
algorithms were re-written in Python, however upon testing it was found that
their performance was significantly slower than the C implementations. This is
most likely a large number of array passing and manipulations are required,
which is significantly slower in Python where function arguments are passed as
object reference instead of pointers in C. Therefore the original C control
algorithms were compiled as a dynamic link library (DLL) and the Python ``C
Types'' library was used to load the DLL and create Python wrappers for the
control algorithms. This significantly improved the performance as all of the
resource intensive calculations were now performed with a compiled C program.

The main control loop that determines the AGC's behaviour is shown below in Fig.
(\ref{fig:control_loop}).
\begin{figure}[H]
\begin{mdframed}
    \begin{center}
        \includegraphics[width=0.9\textwidth]{control_loop.png}
    \end{center}
\end{mdframed}
\captionof{figure}{Main AGC control loop}
\label{fig:control_loop}
\end{figure}
\subsubsection{Map Processing}
As discussed in the introduction to section \ref{software}, all GPS coordinates
are normalised to a power of two. We also ``pad'' the hazard zone coordinates,
before normalisation, by a safety factor of two meters, preventing innacurate
GPS readings or hazard zone coordinates resulting in the AGC entering a hazard
zone. We ``pad'' the hazard zones by calculating the coordinates of centroid of
the polygon, then extending the vector from the centroid to each perimeter point
by two meters. The centroid of a polygon can be calculated from the coordinates
of it's perimeter points as shown in Eqs. (\ref{eq:cx} and \ref{eq:cy}).
\begin{center}
    \begin{equation*}
        A=\frac{1}{2}\sum_{i=0}^{n-1}\left( x_i y_{i+1} - x_{i+1} y_i\right)
    \end{equation*}
\end{center}
\begin{center}
    \begin{equation}
        C_x=\frac{1}{6A}\sum_{i=0}^{n-1} (x_i + x_{i+1}) (x_i y_{i+1} - x_{i+1} y_i)
        \label{eq:cx}
    \end{equation}
\end{center}
\begin{center}
    \begin{equation}
        C_y=\frac{1}{6A}\sum_{i=0}^{n-1} (y_i + y_{i+1}) (x_i y_{i+1} - x_{i+1} y_i)
        \label{eq:cy}
    \end{equation}
\end{center}
\subsubsection{Ray Casting}
The ray casting algorithm is used to determine whether a point falls within a
polygon. The algorithm can only accept polygons with straight edges; our
polygons are represented by a series of GPS points along their perimeter, and so
our polygons are defined by a series of small straight edges. The concept of
this algorithm is that given a point you wish to test, if you cast a ray from
the point in one direction to infinity, the number of intersections the ray has
with the polygon in question indicates whether the point falls in the polygon or
not. If an even number of intersections are found, then the point lies outside
the polygon, and if an odd number of intersections are found the point lies
within the polygon. A visual of this concept can be seen below in Fig.
(\ref{fig:raycasting}).
\begin{figure}[H]
    \begin{mdframed}
        \begin{center}
            \includegraphics[width=0.95\textwidth]{raycasting.png}
        \end{center}
    \end{mdframed}
    \captionof{figure}{Raycasting Example}
    \label{fig:raycasting}
\end{figure}

The figure shows two points which are to be tested to determine if they lie
within the polygon. Casting the ray to the right, towards $x\rightarrow\inf$ in
our implementation, we can count the intersections each ray has with the
polygon. Point A has five intersections and so lies outside the polygon, point B
has four intersections and lies within the polygon. 

There are additional considerations in our implementation to accout for the ray
intersecting a vertex or for the ray being collinear with a polygon edge. If a
vertex is intersected then two intersections are counted, one for each edge
forming the vertex, and the result is not affected. If the ray is collinear with
a polygon edge then only a single intersection is counted, but the ray will
undoubtedly also intersect at least one of the vertices on the collinear edge
which will be appropriately accounted for. The algorithm has been tested and
does not fail in either of these cases.

\subsubsection{Segment Polygon Intersection}
Before the AGC starts moving towards any target destination, it first checks
that it's intended translation vector joining it's location to the target
location does not pass through any hazard zones. It does this by checking that
it's intended translation vector doesn't intersect any edges of polygons. The
diagram shown below in Fig. (\ref{fig:segment_intersection}) depicts two finite
length line segments intersecting.
\begin{figure}[H]
    \begin{mdframed}
        \begin{center}
            \includegraphics[width=0.95\textwidth]{segment_intersection.png}
        \end{center}
    \end{mdframed}
    \captionof{figure}{Line Segment Intersection Diagram}
    \label{fig:segment_intersection}
\end{figure}
 
The edges of polygons are defined only by the positions of two vertices, as is
the AGC intended translation vector. To calculate where the intersection point
of these two line segments are a mathematical equation for a line must be
constructed for each segment as shown in Eqs. (\ref{eq:line_a} and
\ref{eq:line_b}) respectively, then these equations can be solved simultaneously
to determine the location of the intersection as shown in Fig.
(\ref{fig:segment_calculations}).

After the intersection point is calculated we ensure that the intersection point
falls within the smallest rectangle bounded by one point from the first and
second coordinates of either segment. We do this because the equation for a line
is infinite, and any lines with non equal gradient will intersect, but we are
only interested if the intersection occurs between points the AGC intends to
travel. The bounding rectangle is demonstrated by the green rectangle in Fig.
(\ref{fig:segment_intersection}).

\begin{figure}[H]
    \begin{mdframed}
        \begin{center}
            \begin{equation}
                ay_1-y = \frac{ay_2 - ay_1}{ax_2 - ax_1} \times (x - ax_1)
                \label{eq:line_a}
            \end{equation}
            \begin{equation}
                by_1-y = \frac{by_2 - by_1}{bx_2 - bx_1} \times (x - bx_1)
                \label{eq:line_b}
            \end{equation}
        \end{center}
    In our implementation, before this system is solved, we check if the lines
    have equal gradient, and if they are collinear or not. If they have equal
    graident, defined by:
    \begin{center}
        \begin{equation*}
            \frac{ay_2 - ay_1}{ax_2 - ax_1} = \frac{by_2 - by_1}{bx_2 - bx_1}
        \end{equation*}
    \end{center}
    Then we check if they are collinear, defined by:
    \begin{center}
        \begin{equation*}
            (x - ax_1) = (x - bx_1)
        \end{equation*}
    \end{center}
    If they are collinear then they intersect, if the have equal gradient but
    are not collinear then they do not intersect. If one or both of the lines
    are vertical, $(y_1 = y_2)$ then the system is not solved as below but
    trivially using the $x$ position of the line.\vspace{0.5cm}
    \newline
    For simplification:
    \begin{center}
        \begin{equation*}
            m_a = \frac{ay_2 - ay_1}{ax_2 - ax_1} \hspace{1cm} m_b = \frac{by_2 - by_1}{bx_2 - bx_1}
        \end{equation*}
    \end{center}
    Now we solve the system defined by Eqs. (\ref{eq:line_a} and
    \ref{eq:line_b}) to determine the point of intersection resulting in the $x$
    coordinate given by Eq. (\ref{eq:x}) and $y$ coordinate given by Eq.
    (\ref{eq:y}).
    \begin{center}
        \begin{equation}
            x = \frac{ax_{1} m_{a} - ay_{1} - bx_{1} m_{b} + by_{1}}{m_{a} - m_{b}}
            \label{eq:x}
        \end{equation}
        \begin{equation}
            y = \frac{ax_{1} m_{a} m_{b} - ay_{1} m_{b} - bx_{1} m_{a} m_{b} + by_{1} m_{a}}{m_{a} - m_{b}}
            \label{eq:y}
        \end{equation}
    \end{center}
    \end{mdframed}
    \captionof{figure}{Calculation of Line Segment Intersection Point}
    \label{fig:segment_calculations}
\end{figure}

\subsubsection{Pathfinding}
The pathfinding algorithm is only used if the control loop shown in Fig.
(\ref{fig:control_loop}) determines that the AGC should move to the golfer but
the a direct route is not possible due to the presence of a hazard zone between
them. Initially the A* pathfinding algorithm was researched as it is a well
known algorithm with which we could choose some heuristic function to minimise,
in our case that heuristic function would likely be the shortest distance to
reach the golfer. A* is generally used for pathfinding on a discrete graph, we
do not have a discrete graph available and so would have to construct one for
each golf course or a local graph surrounding the hazard we wish to avoid. This
adds unnecessary complexity in terms of implementation and computation to our
system which we wish to avoid. Additionally, A* has time complexity of $O(b^d)$
if $d$ is the shortest path length and b is the branching factor, and the
algorithm has high memory requirements as all generated nodes are kept in memory
\cite{astar_2009}. It was decided we would not use A* algorithm. The next
consideration was to simply ``pad'' the perimeter of the hazard polygon and
follow it all the way around until the AGC reaches the golfer. This concept is
shown in the diagram below in Fig. (\ref{fig:padding}) where the red line
denotes the path that would be taken by the AGC to avoid a hazard zone. Clearly
this is an inefficient path the AGC will unnecessarily follow the contour of the
polygon. Additionally, this method increases the risk that the AGC enters the
hazard zone if our GPS system is not performing well, or if our GPS mapping of
hazard zones is erronous.  


\begin{figure}[H]
    \begin{mdframed}
        \begin{center}
            \includegraphics[width=0.95\textwidth]{padding.png}
        \end{center}
    \end{mdframed}
    \captionof{figure}{Path derived by ``padding'' polygon}
    \label{fig:padding}
\end{figure}

A custom pathfinding algorithm was designed and implemented. The path it finds
is not the optimal solution, but it computes quickly and the path derived is
more efficient than using the ``padding'' method. Fig.
(\ref{fig:pathfinding_flow}) shows the main steps performed by the algorithm to
determine where the AGC should go.

\begin{figure}[H]
    \begin{mdframed}
        \begin{center}
            \includegraphics[width=0.95\textwidth]{pathfinding_flow.png}
        \end{center}
    \end{mdframed}
    \captionof{figure}{Pathfinding algorithm main steps}
    \label{fig:pathfinding_flow}
\end{figure}

First the chord vector, $\overrightarrow{\mathbf{C}}$, is calculated, along with
the angle, $\theta$, between $\overrightarrow{\mathbf{C}}$ and the Y-unit
vector, $\hat{\mathbf{y}}$, using Eq. ().  The relevant polygon, golfer
position, and AGC position are rotated about the z-axis with the center of
rotation as the origin by $\theta$ using Eq. () where $\overrightarrow{p}$ is a
three dimensional point and $\overrightarrow{p^\prime}$ is the rotated point.
The polygon is rotated by applying Eq. () to each vertex. The purpose of the
rotation was to aid the design process of the algorithm and make the code more
readable, removing the necesscity for vector manipulations throuhout the
algorithm instead using only simple arithmetic operations. This results in
$\overrightarrow{\mathbf{C}^\prime}$ having constant $y$. The value of
$y^\prime$ is increment in one meter steps until no intersection between the
translated chord vector, $\overrightarrow{\mathbf{C}^\prime_t}$, and the rotated
polygon, $\mathbf{P}^\prime$. If any point of
$\overrightarrow{\mathbf{C}^\prime_t}$ lies within the polygon the increment
direction is reversed and incremented until there is no intersection and no
points of $\overrightarrow{\mathbf{C}^\prime_t}$ lie within the polygon. This
increment process is repeated in the $x^\prime$ direction for the vector joining
the rotated AGC position, $\left(x^\prime_{AGC}, y^\prime_{AGC}\right)$, to
$\left(x^\prime_{AGC}, y^\prime_{AGC} + y^\prime_i\right)$. Once the
incrementation process if finished such that no intersections are found, the
vector joining $\left(x^\prime_{AGC}, y^\prime_{AGC}\right)$ to
$\left(x^\prime_{AGC}+x^\prime_i, y^\prime_{AGC} + y^\prime_i\right)$ is checked
for intersections. If none are found then $\left(x^\prime_{AGC}+x^\prime_i,
y^\prime_{AGC} + y^\prime_i\right)$ is rotated back to the original coordinate
system and set as the position target for the AGC. If an intersection is found
then $\left(x^\prime_{AGC}+x^\prime_i, y^\prime_{AGC}\right)$ is rotated back to
the original coordinate system and set as the AGC position target. This process
repeats until the golfer is reached. Incrementing in the $y^\prime$ direction is
equivelant to incrementing perpendicular to $\overrightarrow{\mathbf{C}}$ in the
original coordinate system, and incrementing it by $x^\prime$ is equivelant to
incrementing in the direction of $\overrightarrow{\mathbf{C}}$.
\subsubsection{Pathfinding Diagram}
Fig. (\ref{fig:pathfinding_visual}) shows a visual of how the pathfinding
algorithm will determine intermediary positions for the AGC to move it towards
the golfer. For the generation of this figure the pathfinding algorithm was only
called again after the target position is reached, resulting in the very wide
path that can be seen in the final path diagram. On the AGC the pathfinding
algorithm is called every time new GPS data is available, approximately every
second, resulting in a smoother and closer path being found.
\end{multicols}
\newpage
\begin{figure}[H]
    \begin{mdframed}
        \begin{center}
            \begin{minipage}{0.3\textwidth}
                \includegraphics[width=0.95\textwidth]{initial.png}
            \end{minipage}
            \begin{minipage}{0.3\textwidth}
                \includegraphics[width=0.95\textwidth]{p1.png}
            \end{minipage}
            \begin{minipage}{0.3\textwidth}
                \includegraphics[width=0.95\textwidth]{p2.png}
            \end{minipage}
        \end{center}
        \begin{center}
            \begin{minipage}{0.3\textwidth}
                \includegraphics[width=0.95\textwidth]{p3.png}
            \end{minipage}
            \begin{minipage}{0.3\textwidth}
                \includegraphics[width=0.95\textwidth]{p4.png}
            \end{minipage}
            \begin{minipage}{0.3\textwidth}
                \includegraphics[width=0.95\textwidth]{p5.png}
            \end{minipage}
        \end{center}
        \begin{center}
            \begin{minipage}{0.3\textwidth}
                \includegraphics[width=0.95\textwidth]{p6.png}
            \end{minipage}
            \begin{minipage}{0.3\textwidth}
                \includegraphics[width=0.95\textwidth]{p7.png}
            \end{minipage}
            \begin{minipage}{0.3\textwidth}
                \includegraphics[width=0.95\textwidth]{final.png}
            \end{minipage}
        \end{center}
    \end{mdframed}
    \captionof{figure}{Visualisation of path determined by pathfinding algorithm}
    \label{fig:pathfinding_visual}
\end{figure}
\newpage

\begin{multicols}{3}

\subsection{Communications}
The EMG30 motors we are using are controlled by the MD25 motor controller board
via and $I^2C$ interface. The communication interface for this was written in C
and is capable of handling multiple $I^2C$ devices. Currently the only device we
have on the communications network is the MD25. The GPS in the golfer's pod
communicates to the main Raspberry Pi via Bluetooth, which is written in Python
using the PyBluez library. The GPSs communicate with the Raspberry Pis via
serial commuication implemented using the Pyserial library.

\subsection{GCP Elevation API}
The elevation is needed because the height difference in golf shots is relevant
to decide the optimal club choice. Hence, the first idea was to use a Google
Cloud Platform (GCP) elevation application programming interface (API). The
platform allows a free trial period and after this the price per 1000 request is
of 4-5 USD. 

Having the location of the pod, which will be carried by the user, and the
location of the hole in the golf course, the height difference can be known by
transforming the location into latitude and longitude coordinates.


\subsection{GUI}

A GUI is developed to ease communication between the golfer and the caddy. We
installed a screen which is compatible with the Raspberry Pi which acts as the
brain of the AGC. The screen is intended to show displays of the map and enables
the golfer to interact with the AGC for the golf selections, i.e. the golfer can
choose to accept the AGC`s recommended golf club or decline it. The GUI is
developed using PyQt which is a Python binding of the cross-platform GUI toolkit
Qt, used as a Python module. 

 

PyQt is mainly used for creating GUI, developing desktop applications with
Python. As an initial design implementation, we decided to use PyQt as the GUI
of the AGC as the screen is able to showcase the GUI as an application through
the Raspberry Pi. The main reason PyQt is used for this project is due to it
being open source and provides many classes, methods and widgets. This allows us
to make a rich app at zero cost.  

\subsubsection{Course Map}
The course map was implemented on the GUI using the map system that was used on
the simulator. We attempted to embed a map web app called folium which looks
better than the map developed for the simulator, however, it doesn't allow us to
retreive coordinates for a point clicked on the map in real latitude / longitude
coordinates or in screen coordinates. This shortfall makes it impossible to give
the golfer the ability to click a point on the map and see how far away it is
from the AGC, using the simulator map allows us to add that feature.

\section{Course Mapping}
The coordinates of the golf map is required in order for the AGC to understand
its movement boundaries on the golf course. The idea is for the AGC to know the
coordinates of points where it should avoid such as bunkers, lakes and greens
(hazards) while following the golfer throughout the golf course. The coordinates
of the hazards are generated through a software called Mission Planner by
drawing out polygons containing the coordinates of the hazards. Subsequently,
these coordinates are extracted and converted into longitude and latitudes which
are then used for the golf mapping.

Mission planner is created by Michael Oborne and has some features that are
beneficial to this project: 

\begin{itemize}
    \item Point-and-click waypoint, using Google Maps/Bing/Open street maps 
    \item Extraction of coordinates polygons as text file enables flexibility on
    how the data is used 
\end{itemize}

It is used mostly as a ground control station for Plane, Copter and Rover [2].
In this project, its usefulness is its point-and-click waypoint feature which
includes its ability to extract these coordinates. A waypoint is created by
clicking on the map and a polygon is subsequently created. This polygon would
have a unique coordinate (longitude and latitude) based on actual satellite
mapping. The longitude and latitude can be extracted and saved as a text file
which is the intention of this project enabling the coordinates to be
implemented in other algorithms, mainly, the golf mapping and the following
ability of the AGC. 

\begin{figure}[H]
    \begin{center}
        \includegraphics[width=0.3\textwidth]{Municipal.png}
        \captionof{figure}{Southampton Municipal Golf Course on Mission Planner}
        \label{fig:municipal}
    \end{center}
\end{figure}

For testing, the Southampton Municipal Golf Course shown in Fig.
(\ref{fig:municipal}) is chosen due to practical reasons: 

\begin{itemize}
    \item Closest and most accessible golf course for the team to go for testing
    which enables frequent testing at a low cost
    \item Appropriate size of golf course for initial testing which includes a
    perfect amount of hazards preventing overwhelming obstacles during testing
    and design 
\end{itemize}

All the holes and hazards are identified and cross-checked with a golf course
app, VPAR before drawing out the polygons. The waypoints are then created by
drawing polygons around a parameter as in Fig. (\ref{fig:poly2}). For increased
accuracy, the waypoints are drawn as closed as possible to each other as shown
in Fig. (\ref{fig:poly1}).

\begin{figure}[H]
    \begin{center}
        \includegraphics[width=0.3\textwidth]{polygon 2.png}
        \captionof{figure}{Drawing the polygons to define the waypoints for the hazards}
        \label{fig:poly2}
    \end{center}
\end{figure}

\begin{figure}[H]
    \begin{center}
        \includegraphics[width=0.3\textwidth]{polygon 1.png}
        \captionof{figure}{ Increase accuracy of coordinate extraction by having the polygons drawn more closely to each other}
        \label{fig:poly1}
    \end{center}
\end{figure}

\section{Machine Learning}
The AGC will have the option to enable optimal club choice predictions if the
user desires it. This option is placed on the AGC because current golf
regulations outside the United States do not permit the use of software to help
the golfer.

To determine the appropriate machine learning algorithm we first look at the
type of data we have. The AGC will collect both input and output data. Input
data will be mainly personal information from the user and type of club chosen
while output data will be the result of those chosen parameters. Hence, the
model uses supervised learning. From supervised learning we can do a regression
or classification model. The latter one is most suitable because we are
predicting labels and not quantities.


\section{Electronics}
\label{electronics}
\subsection{AGC On Board Electronics}
The trolley will be having a raspberry pi as a microcontroller, two EMG49
motors, an MD49 motor driver, three stepper motors with three A4988 stepper
drivers, a GPS module, a 24v battery and a touchscreen display.

\subsubsection{Processing Unit}
For the trolley microcontroller, the Raspberry Pi 4B with 4 GB of RAM will be
used. Other affordable alternatives of microcontrollers are Arduino boards.
However, the Raspberry Pi is chosen because of its superiority in processing
power, its easy compatibility with touchscreen displays and in-built Wi-Fi
module allowing us to work with an APIs and remote access the system. The
microcontroller will also have a set of heat sinks plus a fan that will avoid
any overheating problems.


\subsubsection{Motors}
We chose to use the EMG49 motors for our prototype. We chose these motors by
considering the top speed we wish the AGC to able to go, and how fast we want it
to accelerate to that speed on a $10^\circ$ incline. The calculations shown
below in Fig. (\ref{fig:motor_calcs}) show how we calculated the necessary
toruqe and rpm we require from our motors.

\begin{figure}[H]
    \begin{mdframed}
        The top speed we chose for the AGC was 5.0 kilometers per hour,
        equivalent to average walking speed. The wheel radius, $W_r$ of the AGC
        wheels is 0.115 meters. We can calculate RPM using Eq. (\ref{eq:rpm}):
        \begin{center}
            \begin{equation}
                rpm = \frac{7.5 * \frac{1000}{3600}}{2\pi W_r} \cdot 60 \approx 115rpm
                \label{eq:rpm}
            \end{equation}
        \end{center}
        The AGC should be able to accelerate to this speed, $V$, within $t=5$
        seconds on a $10^\circ$ incline. For predicted final mass, $M = 15kg$,
        the accelerating force, $F$, required can be calculated with Eq.
        (\ref{eq:accel}):
        \begin{center}
            \begin{equation}
                F = \frac{V}{t} + M g_0 \sin(10^\circ)\approx 29 N
                \label{eq:accel}
            \end{equation}
        \end{center}
        Finally the torque required from each motor can bel calulated using Eq.
        (\ref{eq:torque}):
        \begin{center}
            \begin{equation}
                T = W_r * F \approx 1.7 Nm
                \label{eq:torque} 
            \end{equation}
        \end{center}
    \end{mdframed}
    \captionof{figure}{Motor requirement calculations}
    \label{fig:motor_calcs}
\end{figure}

According to their datasheet the EMG49 motors can provided a loaded rpm of 122
and a torque of 1.56, close to our requirements set out above. As the torque is
slightly lower than calulated, the AGC will accelerate slightly to top speed on
a $10^\circ$ incline in slightly over five seconds.

Shown in Fig. (\ref{fig{emg49}}), is a 24v motor fully equipped with
encoders and a 49:1 reduction gearbox. It is ideal for medium to large robotic
applications, providing cost effective drive and feedback for the user. It also
includes a standard noise suppression capacitor across the motor windings
\cite{emg49data}.
\begin{figure}[H]
    \begin{center}
    \includegraphics[]{emg49.png}
    \label{fig:emg49}
    \end{center}
\end{figure}

\subsubsection{MD49 Motor Controller}
The MD49 motor driver, as shown in Fig. (\ref{fig:electrical_schematic}), is a
robust serial dual motor driver, designed for use with the EMG49 motors. Its
main features are \cite{md49data}:

\begin{itemize}
    \item Reads motors encoders and provides counts for determining distance traveled and direction of rotation.
    \item Drives two motors with independent or combined control. 
    \item Motor current is readable.
    \item Only 24v is required to power the module.
    \item Variable acceleration and power regulation also included
    \item Protection against short circuit and incorrect voltage. 
    \item An error byte can be read to aid troubleshooting.
\end{itemize}

\subsubsection{Stepper Motors}
The release mechanism requires a low RPM and high torque (minimum of 0.13 Nm)
motor to push harder from rest. DC motors tend to have a more sustained output
but less are harder to initialise. Hence, the decision to use stepper motors.
The ones used and shown in Fig. (\ref{fig:electrical_schematic}) are the NEMA 17
stepper motors. They have a maximum holding torque of 0.53 Nm, a rated voltage
of 12-24V and require 1.5A per phase so 3A per motor. 

\subsubsection{A4988 Stepper Motor Driver}
The A4988, shown in Fig. (\ref{fig:stepper_driver}), is a high-performance
stepper motor driver that allows an easy control of one bipolar stepper motor up
to 2A output current per coil. It comes with a heatsink as it can potentially
reach 150$^{\circ}$ \cite{stepper_driver}.
Its operating voltage lies in the range of 8 to 35V. According to the driver
specification, the stepper motor supply requires a 100µF decoupling capacitor
close to the driver to sustain 4A. This capacitor is shown in Fig.
(\ref{fig:electrical_schematic}).

\begin{figure}[H]
    \begin{center}
    \includegraphics[]{stepper_driver.jpg}
    \captionof{figure}{A4988 Stepper Motor Driver}
    \label{fig:stepper_driver}
    \end{center}
\end{figure}

\subsubsection{GPS Module}
The GPS module called NEO-6M shown in DOFIG [picture of gps module] can sense
locations through the whole world by tracking 22 satellites. It takes 27 seconds
to start accurately, provides GPS data in the form of NMEA sentences, and
requires very little power with only 3.3V at 45mA necessary \cite{neo_gps}.

\begin{figure}[H]
    \begin{center}
        \includegraphics[]{gps_module.png}
        \captionof{figure}{GPS Module}
        \label{fig:gps_module}
    \end{center}
\end{figure}


\subsubsection{Power Supply}
The power for the trolley will be supplied by four packs filled with eight AA
1.5V 3Ah batteries. The configuration consists in two pairs of packs in
parallel, resulting in a final configuration of 16s2p of AA cells. This is shown
is DOFIG [electrical schematic diagram]. As a result, the total supply of
voltage is of 24V with a capacity of 6Ah. Initially, the trolley had EMG30
motors rated at 12V and powered by a 12V 16Ah lithium battery inherited from the
supplied trolley. However, the late upgrade in DC motors forced the group to
change the battery to meet the required 24V the new motors needed. Lacking in
budget, the only alternative was to use 32 AA batteries. During testing we found
that this configuratin was unable to supply enough current to the motors,
resulting in the motors unable to run at maximum speed. As out goal was simply
to demonstrate the AGC abilities, it was not necessary for the AGC to operate at
maximum speed, and were a final product to be manufactured then a more
appropriate solution to power the system will be required.

As shown in Fig. (\ref{fig:electrical_schematic}), the first components after
the battery are a 5A fuse and a switch to turn on/off the whole trolley. These
are wires were soldered directly without a base, so an insulated rubber has been
heat-shrinked to avoid short circuits. After this, a stripboard with a 5V and 5A
voltage regulator, shown in Fig. (\ref{fig:voltage_regulator}),
reduces the voltage to 5V. These 5V power the raspberry pi, the GPS module, the
3 A4988 Vcc logic input and the touchscreen. The stepper and DC motors are
powered with 24V so do not need a regulated voltage. 

\begin{figure}[H]
    \begin{center}
        \includegraphics[]{voltage_regulator}
        \captionof{figure}{Voltage Regulator}
        \label{fig:voltage_regulator}
        \end{center}
\end{figure}

\subsubsection{Touchscreen Display}
The touchscreen used is a 5-inch DSI capacitive touch display. It was chosen
because of its convenient connection to the raspberry pi, which only consist of
a ribbon, and its affordable price. It also brings mounting holes and screws to
fit it in a case as we have done in the AGC shown in DOFIG [a picture from Erwins
section, if possible, showing how the screen fits the case]. 

\subsection{Pod Electronics}
The pod, which is used to make the trolley follow the user via GPS, contains
another Raspberry Pi and a GPS module with a 9V battery.
\subsubsection{Microcontroller}
Again, a Raspberry Pi 4B (RP4B) will be used as a microcontroller thanks to its
in-built Bluetooth module which easily allows a connection between two raspberry
pi’s within a range of 240m, allowing the user to separate from the caddie no
more than this distance \cite{raspi}.
The size of the pod can be decreased by using a Raspberry Pi Zero as a
microcontroller instead of the RP4B. This is because the Zero version is much
smaller, has the same built-in Bluetooth and Wi-fi range connection and consumes
only 120mA on idle compared to a higher 540mA for the RP4B. As a result, a
smaller battery would be required, decreasing the overall size, and improving
the customer experience since the smaller and more versatile the pod is, the
more comfortable it will be to keep it in the pocket or other place. However,
due to the worldwide microprocessor manufacturing shortage, there has not been
any available stock for a long time, hence, the use of the groups owns Raspberry
Pis. 

\subsubsection{Power Supply}
The power of the pod is supplied by a 9V 0.4Ah battery. It also contains a
switch to turn on/off the whole pod.

\end{multicols}

\begin{figure}
    \begin{center}
    \includegraphics[width=0.8\textwidth]{electrical_schematic.png}
    \captionof{figure}{Electrical Schematic}
    \label{fig:electrical_schematic}
    \end{center}
\end{figure}

\begin{multicols}{3}

some stuff here


\newpage
\nocite{*}
\bibliographystyle{ieeetr}
\bibliography{./ref}
\end{multicols}



\end{document}